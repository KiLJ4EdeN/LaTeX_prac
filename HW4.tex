\documentclass[onecolumn,12pt]{article}
\usepackage[affil-it]{authblk}
\usepackage{color}
\renewcommand{\thesection}{\color{blue}\Roman{section}.}
\renewcommand{\thesubsection}{\color{blue}\thesection.\Roman{subsection}.}
\providecommand{\keywords}[1]{\textbf{\textit{Keywords---}} #1}
\begin{document}
\title{ HW4 }
\author{Abdolkarim Saeedi}
\affil{\color{black}Islamic Azad University, Science and Research Branch}
\date{March 20, 2020}
\maketitle
\pagenumbering{roman}
\newpage
\keywords{Identity Recognition, Electrocardiography, Deterministic Learning, Pattern Recognition, Ensemble Learning}
\section{\large \color{blue}INTRODUCTION}
\textbf{{\LARGE T}HERE} is a growing consenus that identity has potential applications in many areas to meet security needs. Code locks are one of the most common security systems in everyday life. However, its level of security is very limited because its easily cracked, even with introducing variable password policies (such as mixed case, regularly scheduled password changes) \cite{citation1}. To achieve a higher level of safety, safety systems based on human biometrics (e.g., fingerprints, faces, irises, hand geometry) and behavorial features (e.g., gait, keystrokes, handwriting, voice) have been proposed \cite{citation2}. However, these features are either unreliable in recognition accuracy (e.g., gait, keystrokes) or are suspectible to counterfeiting, such as advances in counterfeiting techniques (e.g., fingerprints can be easily obtained and reproduced by latex; face is sensetive to artificial disguise) \cite{citation3}, \cite{citation4}. As none of these modalities can meet the requirements of all applications, the search for new biometrics goes on. ECG is the recording of the heart's electrical activity measured from the body surface. It comprises three main components: P wave, QRS complex and T wave. The QRS complex is the most unique of them and is less sensetive to physical and emotional changes than other parts of the ECG signal \cite{citation3}. Compared to other biological features as face, fingers, gait, iris, palm, print, sound, etc., ECG has proven to be the most promising biometric, outstanding in the most charactristics that define biometric qualities \cite{citation5}, \cite{citation6}. In the last years, ECG-based biometrics has attracted much attention from the scientific community, and a lot of studies have been conducted \cite{citation7, citation8, citation9, citation10, citation11, citation12}. Moreover, preliminary progress has been made in commercial applications \cite{citation13}, \cite{citation14}. ECG-based identity recognition can go back to the pioneering works \cite{citation15, citation16, citation17}. The main hypothesis shared by these studies is that the detain electrical activity of the heart captured by the ECG signal is sufficient for a high performance human identification system \cite{citation18}. ECG has the following characteristics that other biological features do not have: i) it is difficult to forge and can only be measured from living individuals \cite{citation18}, \cite{citation19}; ii) it has good stability and reproductibility \cite{citation15}, \cite{citation20}; iii) it contains information related to psychological , physiological, and clinical status, which may be of intrest for certain applications \cite{citation18}, \cite{citation21}. In ECG-based identity recognition, the primary problem is to extract features that can realistically depict ECG signals. We can roughly devide the features usdd in the existing research into the following categories: i) Fiducial features that derived from charactristic points of ECG signals \cite{citation15}, \cite{citation16}, \cite{citation19}, \cite{citation21}, such as the temporal intervals and amplitude differences between these characteristic points (e.g., the RR interval); ii) Non-fiducial features derived from segmented windows of ECG signals \cite{citation22, citation23, citation24} without fiducial points (or only with the R peaks). Principal components, wavelet coefficients, and autocorrelation coefficients are some examples; iii) Hybrid features that combine fiducial features with no-fiducial features to create the feature set \cite{citation3}, \cite{citation25}. However, these static features cannot characterize ECG signals adequately, since ECG signals are essentialy temporal patterns (i.e., time-varying patterns) with significant variations \cite{citation26}, \cite{citation27}. From the perspective of patter recognition, identity recognition based on ECG signals is essentialy a dynamic pattern recognition problem. The existing classification methods developed for static patterns are probably not suitble for ECG-based human identification \cite{citation27}. It has been pointed out in \cite{citation28}, \cite{citation29} that the temporal pattern recognition should be fundamentally different from the static pattern recognition method. Recently, a new algorithm, deterministic learning theory, has been proposed to deal with temporal patterns \cite{citation29, citation30, citation31}. These dynamics of the temporal patterns can be accurately modeled and expressed in a time-independent manner by using deterministic learning. In particular, these representation contains complete information of the temporal pattern. Therefore, it may be more suitable for recognizing temporal patterns with large morphological changes and difficult to be accurately characterized by static features (e.g., electrocardiogram, electroencephalogram). Based on the representation, Wang and Hill proposed a temporal pattern similarity measurement in \cite{citation31}.\\In this paper, we propose a new ECG-based method for identity recognition. Considering that QRS complexes are not susceptible to physical activity and emotional states, and provide the most relevant and unique information in ECG signals \cite{citation10}, \cite{citation32}, this article intends to conduct identity recognition based on QRS complex. First we devide all ECG signals into train and test sets. Second we extract the QRS conplex of all the ECG signals in the training set, and accurately model them based on deterministic learning theory, and store the QRS modeling results and corresponding identity tags to a pattern library. For the ECG signals in the test set, we first classifiy their QRS complexes and them classify them by voting based on the classification results of the QRS complexes.
\begin{thebibliography}{999}
\bibitem{citation1}
Author not decarled,
\emph{\LaTeX: No infornation on this article}.
\bibitem{citation2}
Author not decarled,
\emph{\LaTeX: No infornation on this article}.
\bibitem{citation3}
Author not decarled,
\emph{\LaTeX: No infornation on this article}.
\bibitem{citation4}
Author not decarled,
\emph{\LaTeX: No infornation on this article}.
\bibitem{citation5}
Author not decarled,
\emph{\LaTeX: No infornation on this article}.
\bibitem{citation6}
Author not decarled,
\emph{\LaTeX: No infornation on this article}.
\bibitem{citation7}
Author not decarled,
\emph{\LaTeX: No infornation on this article}.
\bibitem{citation8}
Author not decarled,
\emph{\LaTeX: No infornation on this article}.
\bibitem{citation9}
Author not decarled,
\emph{\LaTeX: No infornation on this article}.
\bibitem{citation10}
Author not decarled,
\emph{\LaTeX: No infornation on this article}.
\bibitem{citation11}
Author not decarled,
\emph{\LaTeX: No infornation on this article}.
\bibitem{citation12}
Author not decarled,
\emph{\LaTeX: No infornation on this article}.
\bibitem{citation13}
Author not decarled,
\emph{\LaTeX: No infornation on this article}.
\bibitem{citation14}
Author not decarled,
\emph{\LaTeX: No infornation on this article}.
\bibitem{citation15}
Author not decarled,
\emph{\LaTeX: No infornation on this article}.
\bibitem{citation16}
Author not decarled,
\emph{\LaTeX: No infornation on this article}.
\bibitem{citation17}
Author not decarled,
\emph{\LaTeX: No infornation on this article}.
\bibitem{citation18}
Author not decarled,
\emph{\LaTeX: No infornation on this article}.
\bibitem{citation19}
Author not decarled,
\emph{\LaTeX: No infornation on this article}.
\bibitem{citation20}
Author not decarled,
\emph{\LaTeX: No infornation on this article}.
\bibitem{citation21}
Author not decarled,
\emph{\LaTeX: No infornation on this article}.
\bibitem{citation22}
Author not decarled,
\emph{\LaTeX: No infornation on this article}.
\bibitem{citation23}
Author not decarled,
\emph{\LaTeX: No infornation on this article}.
\bibitem{citation24}
Author not decarled,
\emph{\LaTeX: No infornation on this article}.
\bibitem{citation25}
Author not decarled,
\emph{\LaTeX: No infornation on this article}.
\bibitem{citation26}
Author not decarled,
\emph{\LaTeX: No infornation on this article}.
\bibitem{citation27}
Author not decarled,
\emph{\LaTeX: No infornation on this article}.
\bibitem{citation28}
Author not decarled,
\emph{\LaTeX: No infornation on this article}.
\bibitem{citation29}
Author not decarled,
\emph{\LaTeX: No infornation on this article}.
\bibitem{citation30}
Author not decarled,
\emph{\LaTeX: No infornation on this article}.
\bibitem{citation31}
Author not decarled,
\emph{\LaTeX: No infornation on this article}.
\bibitem{citation32}
Author not decarled,
\emph{\LaTeX: No information on this article}.
\end{thebibliography}
\end{document}