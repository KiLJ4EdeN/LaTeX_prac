\documentclass[a4paper,12pt]{article}
\usepackage[affil-it]{authblk}
\usepackage{color}
\usepackage{hyperref}
\begin{document}
\title{Writing and publishing a scientific paper}
\author{Abdolkarim Saeedi}
\affil{\color{black}Islamic Azad University, Science and Research Branch }
\date{\today}
\maketitle
\begin{abstract}
How to write a scientific paper? How to publish it? This list of notes contains tips drafting and writing a paper, on submitting it, and on dealing with referees and editors. I write this from the perspective of an author, but more importantly from the perspective of reviewer and editor: I regularly read scientific papers that could be greatly improved. The current document was suggested by a number of readers of an early set of advice: How to write a thesis.
\end{abstract}
\newpage
\tableofcontents
\newpage
\section{To do before you even think of papers}
What question are you asking and answering? Why? Who would be interested?\\\\Do bring these questions to the forefront of your attention from time to time. It's possible to become involved in the investigation to such a point that you lose sight of the main questions. You ought to be capable of answering the first one to yourself. It's a good idea to imagine answering the second and third to a (potentially difficult) reviewer or editor.\\\\The materials and methods section of an experimental paper will be based on your lab journal. It will possibly include diagrams based on your lab journal. So for these, as well as ethical, practical and legal reasons, \textbf{keep your journal carefully}. Record everything that seems relevant (and even things that don't, if it's not too hard). It’s a good idea, when designing an experiment or tinkering with a technique to ask yourself: “how will this look in the materials and methods section?” or "it seems reasonable to me, but will it convince a skeptical reviewer?".\\\\\textbf{Analyse}. For all your work, look at everything carefully. From time to time, discuss your work with a specialist and non-specialist. Look for new angles, new questions or answers that your data might be telling you. Try to argue new cases. Especially, argue against your own hypothesis or interpretation (because somebody else may do so, and you should be prepared). Does the design of the investigation need improvement? And finally, what do the data really say? It may not be what you think!
\section{Writing a paper}
Starting with the blank paper or screen is often the hard part. Fortunately, you have a story to tell, or you soon will have. First, ask yourself “What is this about?”,“What is important?” (Not what is important to you – to other people!)  The answer to that is the raw material for the beginning of your story, about which we’ll talk further later, under subheadings.\\\\It’s a good idea to start to assemble your argument, in brief form, right from the beginning. Perhaps words and bullet points  are enough, but often you will need graphs and diagrams.  Consider using diagrams for arguments, especially complicated arguments. They will not usually appear in the final version, but they could be useful to you in assembling the points in your story. I think that writing text is not a good way to begin.\\\\My method is always to start with the figures that will appear in the Results section. The figures need not be in final form, indeed a sketch will be enough. Then I try assembling them in different orders, to find which is the most logical. Then I imagine telling the story, using these pictures. This is the key step. If I have a coauthor, I then tell it to him/her and we discuss the order and refine it. Once you have refined the order of your argument, the difficult sections of the paper (introduction, results and discussion) become easier to write. Again, outline your arguments in point form before you start writing text.\\\\
\section{Which journal}
You need to find a readership who will be interested in what you have done (or should be!). Who are they? What journals would they read regularly?\\\\A good guide is to look at the papers that you will cite – papers that are relevant to your work. In what journals did they appear? Editors may ask the same question about your paper. It is true that the editor might like to see citations to his/her own journal for other reasons, but s/he is also likely to ask the question sincerely: if you haven’t referred to any work from this journal, what makes you think its readers will be interested in your work?\\\\Before you decide, you should ask whether the topic really is in the ambit of that journal. One question that the reviewers are likely to be asked is how well your paper fits. Finally, when you have chosen, have a look at the nomenclature, styles, protocols and format of the journal before you start writing. Be aware that sending it to the wrong journal may slow your publication by months. 
\section{The introduction}
You should start with a brief section, perhaps a paragraph, in which you narrow the reader’s attention from the discipline of the journal down to your story. Essentially, you should answer the questions: in the field of …., why is your question an important or interesting one? Some journals explicitly invite authors to end the first paragraph with a sentence that starts 'In this study, ... '\\\\Follow this with a fairly brief discussion of the necessary background. You should refer to the relevant literature (not only because it's important, but also out of collegial respect). However, you’re not (usually) writing a review and the readers of this journal, especially those who have chosen to read your paper, are not complete novices. It’s hard to say how long this should be: from one to several paragraphs.\\\\Then finish with a paragraph or two on why it was interesting and important to do your study and, very briefly, why you chose to do it this way. 
\section{Materials and methods}
This should be easy to write, especially if your lab journal is well documented. You need to explain what you did in such a way that a competent worker in the field could reproduce it, what apparatus and materials you used and, in nonobvious cases, where you obtained them. Calibration might go here, unless it is either trivial and standard (so omit) or novel and important, in which case it could be part of the result. A well-made schematic diagram is usually more helpful than a photograph of the apparatus.
\section{Results and discussion}
Should the results and discussion be separate or a single section? The answer depends on your case. Using separate sections has the advantage that the reader can look through the results and try to understand them, without being biased by your explanation. It has the disadvantage, especially if there are many parts to each section, that one either has to repeat material, or else ask the reader to skip backwards and forwards between sections.\\\\Often you can let graphs tell the story. Spend some time plotting your results – perhaps in several different ways. Sometimes extra analysis and a different plot will show you things that you missed before. Take some time and think hard. Further, a really well planned presentation in a figure may make the difference between a reader understanding it or not. In most cases, putting the legend in the figure rather than the caption, makes it easier for readers to understand.\\\\Once you have a set of figures, even in draft form, assemble them and use them as evidence as you tell a story to a real or imagined interlocuter. Imagine someone who doesn’t understand and who won’t be easily convinced. When you’ve done this, you have the outline of your story.
\section{Conclusions}
Often the conclusions form a separate section, but they may be part of the discussion. Ultimately  you should answer your own questions – at least in part. If you have a conclusions section, keep it to conclusions. It is not a summary (that's your abstract). And it's not a second chance to discuss the results (that's the Discussion section).
\section{Abstract}
As a reviewer and editor, I am weary of reading abstracts that are largely introduction. The introduction is for that. Although the abstract appears at the beginning, you should write the abstract \textit{last}. It is \textit{not} an introduction. It is a résumé. Here you tell the reader – including all of the many readers who will read the abstract but not the paper – what you actually show in this paper. Try to be quantitative.
\section{Style}
Research papers are written in a relatively formal style – more formal than this document. Remember that many researchers speak English as a second or third language: they will find it easier if you keep to short sentences and avoid rare words and informal or innovative constructions.\\\\An excellent and widely used reference for English grammar and style is \textit{A Dictionary of Modern English Usage} by H.W. Fowler. One plea from this editor: The active voice ("we measured the frequency by…") is simpler than the passive ("the frequency was measured by …"), and it makes clear what you did. It also leads to fewer awkward sentences. (Unfortunately, some journals forbid it.)
\section{References}
How many? No simple answer: some general journals severely limit the number. A review paper may have hundreds. As many as you need, plus often some for politeness: yes, a reference to a recent review or textbook will give the reader the background, but references to the original discoverer not only honours him/her, but also shows that you’ve read the primary sources. This need not be exaggerated: there is no need to cite the \textit{Principia} every time you use Newton's laws of motion.\\\\I’ve mentioned above, when discussing ‘which journal?’, that your reference list will probably include references from the target journal.\\\\And finally, be aware that editors may look at your reference list when searching for suitable reviewers: they are, after all, people doing relevant work in the field.
\section{Figure captions}
Try to keep the captions short, but still allow the information in the figure (though not necessarily the explanation) to be understood on its own. Take advantage of any elegant ways of putting information into legends on the figure, rather than in the caption. 
\section{Figures}
Figures are often the most important part of a scientific paper, so please take some time making really good quality figures.\\
\begin{itemize}
\item Remember that nearly all journals print in monochrome.
\item Look at the journal and see how wide its columns are: that width (or occasionally twice that width) are the preferred size of your figures.
\item It may be be possible to make publication-quality figures using Excel, but I've not yet seen it done: that software was designed for accountants and administrators – a much bigger market than scientists.
\item Figures with symbols differentiated by colour rather than shape, data without error bars, inefficient use of space (and especially figures made using Excel): they all shout “I am not a researcher”, which is probably not the image you wish to project.
\item Schematic diagrams are usually more informative than photographs: they show how it works rather than what it looks like. However, in some cases a well made photograph is useful. Take the trouble to find a plain background and good lighting. 
\end{itemize}
\section{Title and key words}
Think carefully about these: they will be used by indexing services and search engines. They could be important in attracting the relevant readers to your work.
\section{Format}
Read the journal’s \textit{Instructions to authors} carefully.  Then follow it closely, including its referencing style. Either that or annoy the editor – it’s your choice.\\\\By the way, you may think that the editor is a person whose job it is to edit. This is almost never the case. Usually, the editor (like me) is someone whose job is to be a scientist and who has had the editing job added to his/her many other duties, with no time allowance or payment. Even sending the ms back to you, unread, with a covering note saying "Read the Instructions to authors" takes time. 
\section{Covering letter}
Usually, your submitted manuscript will be accompanied by a covering letter. In this you explain why that journal is appropriate for this paper and why its readers would be interested. (A cynic would say that some editors may be influenced if you can imply that publishing your paper will attract a high number of citations, but I doubt that this is important.)
\noindent\rule{\textwidth}{1pt}
\section{Reviews and reviewers} 
 My second paper (and my first as a sole author) came back with a few dozen suggestions for changes. At the time, I was disappointed: it implied that my paper wasn’t perfect! Well, of course it wasn’t. Nowadays, I am disappointed  when there are no suggested changes. (Nearly) every such suggestion is advice from an expert on how you can make your paper better and more understandable. The peer review system offers us free, expert advice – let's be grateful.\\\\If the comments are negative, \textit{cool off}. Here is a well known example of how not to react. If the reviewers didn't understand, it's usually your fault, to a large extent. Yes, they read it quickly, but they are experts in the field. If they didn't understand it in a quick read, it usually means that you've not explained it well. So think "How could s/he have thought that and why?" Put yourself in their position – they haven't been as close to this problem as you, and they (like most readers) are not as interested in it as you are.\\\\Yes, it’s probably taken them a while to get back to you. But there is the advantage that now you can read your paper again, with a fresh view.\\\\From my perspective as an editor, the system of peer review seems to work very well. Consider the proportional benefit: A scientific paper typically represents an amount of work on the order of a person-year: a very big investment. If a good reviewer takes several hours to read the paper and to give explicit suggestions on how it can be improved and made easier to understand, then that represents an important and very efficient contribution. The problem is that, because of anonymity (which has other important advantages), the reviewers don't receive public thanks for that important input. So I hope that your paper receives such benefits: identification of errors and suggestions that will make the next iteration easier to understand (and less easy to misunderstand). 
\section{But what if they reject it?}
Console yourself that good papers are sometimes rejected. Researchers often agree that a 'small step' paper that fits into the established ideas is easier to publish than is a completely new idea. However, you should read the reasons given for rejection and consider very seriously the possibility that the editor and reviewers either are right, or else have misunderstood your work.\\\\Sometimes, you may be able to rewrite and to resubmit. This usually involves extensive rewriting: if they have misunderstood your paper, then the new version should be nearly impossible to misunderstand.\\\\It is possible to try another journal, where you may or may not get the same reviewers. If you do this, again, make sure that the new version will be understood.\\\\Most importantly, consider the possibility that the reviewers are correct.
\section{It's accepted!}
Read the proofs very carefully, especially any equations. By the time the proofs come, the paper will be out of your short term memory, so you can read it again with a fresh pair of eyes.\\\\Regarding copyright: My advice is not to be too generous. As researchers, we create the content for journals, we review them, we edit them and very often we deliver them to the publisher in a form that requires very little effort to publish. Then the publishers, or at least a couple of large multinational ones, bankrupt our libraries by charging exorbitant subscription fees. So don’t be too generous. If there is a line asking you not to post your own paper on your own server, consider crossing it out before you sign, or how you can get around it. (One option might be posting a manuscript version of the final published paper). 
\section{Journals, delays and impact}
The time to publication varies greatly both among journals and among papers in the same journal. Usually, a journal publishes the dates when a paper was received, when it was accepted, and when it appeared on line and/or in hard copy. (The last is not always simple: the nominal date of publication does not always coincide with the appearance in press.) So you can obtain an estimate of how long it will take – which is usually disappointing news to new authors. Surprisingly often, the delay includes a substantial component from the authors: the journal recommends changes and authors take a while to make these, or at least to make them to the satisfaction of the editor. The delay varies very considerably among journals, however, and this is one factor that you may consider to be important.\\\\Different journals have different prestige, popularity, impact and impact factor. This last is defined in a number of different ways, including this one: the number of citations the articles in a journal receive in a given year divided by the number of articles published. Some people use impact factor to obtain a simple quantitative estimate of importance. Some journals, including some of those with high impact factors, are reputed to have a high rate of rejection of submitted papers. The editors may decide that, although a paper is technical correct, it is not important enough to be selected for the limited space that they have.\\\\To take two extreme examples, \textit{Nature} and \textit{Science} are generalist science journals with very high impact factors, however calculated, and they enjoy considerable prestige. They are highly selective and have two filters. The first is the editors’ judgment of the broad interest and importance of a manuscript. A very large majority of (probably very good) papers are rejected at this stage as being insufficiently important and interesting to attract widespread interest from a very broad readership. Only after passing this filter is one’s manuscript sent for review. These journals have the advantage of great speed: you find out in a week or so whether your paper has gone to review and, if it has, you have the reviews within another couple of weeks. The time from submission to publication may be only several weeks, compared with several months or more for most journals. 
\noindent\rule{\textwidth}{1pt}











































































































\end{document}
